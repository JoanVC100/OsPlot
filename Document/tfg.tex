\documentclass{tfgitic}[2023/06/30]
% per a fórmules químiques
\usepackage[version=4]{mhchem}
% Per dibuixar gràfics: base general i gràfics senzills
\usepackage{tikz} \usetikzlibrary{arrows}
% Per dibuixar gràfics: circuits electrònics
\usepackage[europeanresistors,americaninductors]{circuitikz}
% Per dibuixar gràfics: diagrames diversos
\usepackage{pgfplots}
% Per escriure algoritmes
\usepackage[plain,figure]{algorithm2e}
% Per a usar taules elàstiques
\usepackage{tabularx}

% Indica quines bd bibliografiques usarem
\addbibresource{tfg.bib}

% Marges en els algoritmes
\setlength{\algomargin}{4em}

% Versió de pgfplots a usar
\pgfplotsset{compat=newest}

\definecolor{tufte1}{rgb}{0.7,0.7,0.55}
\definecolor{yellowmarker}{HTML}{DEF440}

\pgfplotsset{
	tufte bar/.style={
		ybar,
		axis line style={draw opacity=0},
		xtick=\empty,
		ymin=0,
		bar width=3mm,
		x=2*\pgfkeysvalueof{/pgf/bar width},
		ymajorgrids,
		grid style=white,
		axis on top,
		major tick length=0pt,
		cycle list={
			fill=tufte1, draw=none\\
		},
		enlarge x limits={
			abs=0.5*\pgfkeysvalueof{/pgf/bar width}
		},
		axis x line*=bottom,
		x axis line style={
			draw opacity=1,
			tufte1,
			thick
		},
		yticklabel=\pgfmathprintnumber{\tick}\,\%
	}
}

\title{OsPlot: Com implementar osci\lgem oscopis amb un microcontrolador i Linux}

\subtitle{Subtítol}

\author{Joan Vilardaga Castro}

\advisor{Francisco Del Aguila López}

\dedication{Dedicació}

\begin{acknowledgments}
Tot el meu agraïment.
\end{acknowledgments}

\begin{resum}
Aquest document descriu una solució anomenada OsPlot, una arquitectura
genèrica que permet transformar un microcontrolador i un ordinador en
un osci\lgem oscopi funcional.  Mitjançant l'ús de plataformes obertes
com Linux, Arduino i GNUPlot, aquesta investigació intenta oferir una
solució de baix cost i accessible per a gent interessada en l'anàlisi
de senyals.
\end{resum}

\begin{abstract}
This document describes a solution named OsPlot, which is a generic
architecture that allows transforming a microcontroller and a computer
into a functional oscilloscope. By using open platforms such as Linux,
Arduino and GNUPlot, this research aims to provide a low-cost and
accessible solution for people interested in signal analysis.
\end{abstract}

\begin{document}

\part{Memòria}

\chapter{Introducció}
\label{cap:intro}

\section{Origen del treball: Osci\lgem oscopi amb GNUPlot}
\label{sec:origen}

GNUPlot \cite{gnuplot} és un programa de traçat de gràfics en 2D i 3D
que permet visualitzar i analitzar diferents conjunts de dades. Tot i
que GNUPlot és una eina per fer gràfics estàtics, es pot combinar amb
altres programes o ``scripts'' per aconseguir la funcionalitat de
traçat en temps real.

És precisament aquest descobriment el que ha obert la porta a la
possibilitat de fer un osci\lgem oscopi amb GNUPlot com a plataforma
principal. Així, aquesta investigació s'enfoca a explorar la
viabilitat d'utilitzar GNUPlot per a la creació d'un osci\lgem oscopi
digital totalment funcional.

\section{Objectius}
\label{sec:objectius}

\begin{enumerate}
	\item Evaluar la capacitat de GNUPlot per visualtizar dades en
          temps real.
	\item Desenvolupar i avaluar OsPlot com a solució genèrica per
          transformar un microcontrolador i un ordinador en un
          osci\lgem oscopi.
	\item Implementar programes de referència que explotin les
          diferents possibilitats de l'arquitectura.
\end{enumerate}

\section{Com funciona un osci\lgem oscopi}
\label{sec:com-funciona-oscil·loscopi}

Un osci\lgem oscopi \cite{viqui-oscil·loscopi} és un instrument de
mesura que s'utilitza per visualitzar i analitzar senyals elèctrics
variables en el temps. Com amb moltes altres coses, els osci\lgem
oscopis són instruments que han evolucionat amb la digitalització,
creant una nova generació d'osci\lgem oscopis més precisos i que poden
fer operacions molt més complexes com mitges, pic a pic o
transformades ràpides de Fourier.

\subsection{Funcionament bàsic}
\label{subsec:funcionament-bàsic}

Els osci\lgem oscopis digitals es basen en l'ús d'un ADC (convertidor
analògic-digital), un sistema que converteix senyals analògics en
senyals digitals, les quals es processen per generar una representació
gràfica en temps real a la pantalla. L'eix horitzontal indica el
temps, i l'eix vertical indica generalment voltatge (tot i que es
poden mesurar altres magnituds com el corrent).

Tots els osci\lgem oscopis incorporen un subsistema de ``trigger''
\cite{funcionament-trigger} que té com a funció establir un punt de
referència i sincronitzar la captura de les dades. Aquest subsistema
és essencial per garantir una visualització estable i repetible. Sense
el ``trigger'', el senyal a visualitzar es desplaçaria constantment a
través de l'eix horitzontal, fent impossible la seva observació.

\subsection{Conceptes importants de senyals digitals}
\label{subsec:conceptes-pds}

Com que el treball se centra en osci\lgem oscopis digitals, és
necessari conèixer alguns dels conceptes claus del processament
digital del senyal \cite{llibre-pds}:

\begin{itemize}
	\item Senyal analògic (o continu): Tipus de senyal format per
          variables contínues. Encara que aquests valors estiguin
          confinats en un rang màxim i mínim, poden tenir una precisió
          infinita, no hi ha espais buits.
	\item Senyal digital (o discret): Seqüència de valors discrets
          separats en el temps. A diferència dels analògics, els
          senyals discrets tenen espais buits.
	\item Quantització: És el procés d'aproximar un valor continu
          al valor discret més proper. Sigui aproximació o truncament,
          sol introduir error de mesura, ja que un valor discret no
          pot representar els infinits decimals d'un valor continu.
	\item Resolució: És la diferència entre dos valors d'un rang
          discret. Determina el màxim error de discretització i el
          nombre de bits d'una mostra.
	\item Mostreig: Procés que transforma un senyal analògic en un
          senyal discret mitjançant l'ús repetit de la quantització.
          Cada valor quantitzat és una mostra del senyal analògic.
	\item Freqüència de mostreig: El nombre de mostres que captura
          un ADC cada segon. La màxima freqüència que podrà captar
          l'osci\lgem oscopi serà com a màxim la meitat de la
          freqüència de mostreig (criteri de Nyquist).
\end{itemize}

\printbibliography

\end{document}

%%% Local Variables:
%%% mode: latex
%%% TeX-master: t
%%% LaTeX-biblatex-use-Biber: t
%%% End:
